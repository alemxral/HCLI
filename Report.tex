\documentclass[a4paper,12pt]{article}
\usepackage{graphicx}
\usepackage{hyperref}
\usepackage{listings}
\usepackage{xcolor}
\usepackage{geometry}
\geometry{margin=1in}

\title{Habit Tracker CLI - Technical Report}
\author{Your Name \\ IU University}
\date{\today}

\begin{document}

\maketitle

\section{Introduction}
The Habit Tracker CLI is a command-line application designed to help users efficiently track their habits. The tool enables users to create, check off, and analyze their habits over time, providing meaningful insights into their daily and weekly routines. This document outlines the technical foundations of the project, the design choices made during development, and the final implementation.

\section{Conception Phase}
The conception phase is critical for defining the technical architecture of the Habit Tracker CLI. Below are the key design considerations:

\subsection{Core Components}
\begin{itemize}
    \item \textbf{Habit Class}: Represents an individual habit, including its name and periodicity (daily/weekly).
    \item \textbf{Storage System}: Habits and logs are stored in JSON format to ensure ease of access and modification.
    \item \textbf{Command-Line Interface (CLI)}: Users interact with the system via Typer, a Python framework for building CLI applications.
    \item \textbf{Analytics Module}: Provides statistics on habits, including streaks, reminders, and missed check-ins.
\end{itemize}

\subsection{User Interaction and Workflow}
Users interact with the application through a structured flow:
\begin{enumerate}
    \item The user starts the program and sets up their username.
    \item Habits are added with a name and a periodicity (daily or weekly).
    \item Users check off completed habits, which are logged with timestamps.
    \item The system provides analytics, including habit streaks and reminders for pending tasks.
    \item Users can configure file locations for data storage via the config command.
\end{enumerate}

\subsection{System Architecture}
The following diagram illustrates the core components and their interactions:

\begin{figure}[h]
    \centering
    \includegraphics[width=0.8\textwidth]{uml_diagram.png} % Replace with actual UML diagram
    \caption{UML Diagram of Habit Tracker System}
    \label{fig:uml}
\end{figure}

\section{Development Phase}
During development, the components outlined in the conception phase were implemented using Python and relevant frameworks.

\subsection{Technologies and Tools Used}
\begin{itemize}
    \item \textbf{Python}: Primary programming language.
    \item \textbf{Typer}: CLI framework for managing user commands.
    \item \textbf{Rich}: Provides colored output and formatted tables in the CLI.
    \item \textbf{JSON}: Used to store habit data and user settings.
    \item \textbf{Pytest}: Ensures the correctness of core functionality via unit tests.
    \item \textbf{GitHub}: Version control and project collaboration.
\end{itemize}

\subsection{Implementation Details}
\begin{itemize}
    \item The application consists of modularized classes for easy maintainability.
    \item The data storage mechanism ensures persistence by writing habit data to JSON files.
    \item User interactions are intuitive, following a clear command hierarchy.
    \item Error handling is implemented to avoid crashes and unexpected behavior.
\end{itemize}

\subsection{Testing Strategy}
To ensure the robustness of the application, a comprehensive test suite was developed:
\begin{itemize}
    \item Unit tests verify core functionality such as adding, checking, and deleting habits.
    \item The test suite resets test data before execution to ensure consistency.
    \item Coverage analysis was performed using pytest-cov.
\end{itemize}

\section{Finalization Phase}
After the development phase, the project was refined and prepared for final submission.

\subsection{Repository and Deployment}
The Habit Tracker CLI is hosted on a public GitHub repository:
\begin{itemize}
    \item \textbf{GitHub Link}: \href{https://github.com/alemxral/HHCLI}{https://github.com/alemxral/HHCLI}
    \item The repository contains the complete source code, README, and unit tests.
    \item Users can clone the repository and install dependencies via:
    \begin{lstlisting}
    git clone https://github.com/alemxral/HHCLI.git
    cd HHCLI
    pip install -r requirements.txt
    python main.py --help
    \end{lstlisting}
\end{itemize}

\subsection{Challenges and Lessons Learned}
\begin{itemize}
    \item **Handling User Input:** Initial versions of the CLI did not properly validate input, leading to errors.
    \item **Testing Edge Cases:** Implementing unit tests revealed edge cases that were not considered initially.
    \item **Performance Optimization:** The habit storage system was optimized to improve read/write speeds.
\end{itemize}

\subsection{Future Improvements}
\begin{itemize}
    \item Adding a graphical user interface (GUI) to complement the CLI.
    \item Implementing habit syncing with cloud storage.
    \item Introducing notifications/reminders via email or mobile push notifications.
\end{itemize}

\section{Conclusion}
The Habit Tracker CLI provides a powerful yet lightweight solution for users to manage their habits effectively. The project follows best practices in software development, including modular architecture, error handling, and automated testing. With further enhancements, this tool could be expanded into a fully-featured habit management platform.

\end{document}
