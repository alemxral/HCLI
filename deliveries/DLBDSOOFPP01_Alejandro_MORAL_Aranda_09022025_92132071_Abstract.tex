\documentclass[a4paper,12pt]{article}

% Packages
\usepackage{geometry}   % Page layout
\usepackage{hyperref}   % Hyperlinks
\usepackage{titlesec}   % Formatting section titles

% Page setup
\geometry{left=2.5cm, right=2.5cm, top=2.5cm, bottom=2.5cm}

% Title Formatting
\titleformat{\section}{\Large\bfseries}{\thesection}{1em}{}

\begin{document}

% Title
\begin{center}
    {\LARGE \textbf{Habit Tracker HCLI}}\\[0.5cm]
    {\Large \textbf{Abstract}}\\[0.3cm]
    {\small Author: Alejandro Moral Aranda \hspace{1cm} Date: \today}
    \hrule
\end{center}

\section{Introduction}
The Habit Tracker HCLI is a command-line application that helps users create, manage, and analyze their daily and weekly habits. The motivation behind the project was to provide a lightweight and distraction-free tool for habit tracking, accessible entirely through the terminal. 

Unlike traditional habit tracking applications, HCLI stores data in JSON format for persistence, provides analytics to monitor progress, and enables users to configure storage paths for flexibility.

\section{Implementation Overview}
The application is built using:
\begin{itemize}
    \item \textbf{Python} for core functionality.
    \item \textbf{Typer} for the command-line interface.
    \item \textbf{Rich} for displaying tables and formatted output.
    \item \textbf{Pytest} for unit testing.
\end{itemize}

Key features include:
\begin{itemize}
    \item Adding and managing daily/weekly habits.
    \item Checking off completed habits, including past dates.
    \item Tracking streaks and generating habit analytics.
    \item Displaying dashboards in ASCII and graphical format.
    \item Configuring storage paths for habit/user data.
\end{itemize}

\section{Challenges and Solutions}
\textbf{1. Avoiding Duplicate User Input:}  
Initially, the `setup-user` command asked for input twice. This was resolved by ensuring config file creation before user input.

\textbf{2. Handling Edge Cases in Streak Calculation:}  
Edge cases were identified where incomplete check-ins broke streaks incorrectly. These were fixed by adjusting date validation logic.

\textbf{3. Ensuring Cross-Platform Compatibility:}  
The script was tested on different operating systems to ensure compatibility with Windows, macOS, and Linux terminals.

\section{Lessons Learned}
\begin{itemize}
    \item User Experience in CLI Matters: Clear and structured output using Rich improves usability.
    \item Configuration is Key: Allowing users to modify paths for storing JSON files provides flexibility.
    \item Testing Prevents Bugs: Unit testing with Pytest helped catch issues before deployment.
\end{itemize}

\section{Final Thoughts}
The project successfully met its goal of providing a functional CLI-based habit tracker with analytics and flexible configuration. Future improvements could include:
\begin{itemize}
    \item Cloud storage integration for syncing habits across devices.
    \item Reminder notifications for pending habits.
    \item Enhanced visualization using external dashboards.
\end{itemize}

\section{GitHub Repository}
The complete project, including the source code and documentation, is available on GitHub:  
\url{https://github.com/alemxral/HCLI}

\end{document}
